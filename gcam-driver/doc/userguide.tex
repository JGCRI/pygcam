\documentclass[11pt]{article}
\usepackage{amsmath}

\title{{GCAM} Automation System Users' Guide}

\begin{document}

\maketitle

\section{Overview}
The GCAM automation system is intended to provide a way to run all the
steps of a GCAM analysis using a single command.  Such an analysis
might include several data preprocessing steps, a run of the GCAM core
model, and several data postprocessing steps.  The steps (called
``modules'' throughout this documentation) to be performed are laid
out in a configuration file, the format of which is designed to be
easily edited by hand or by an automated process (such as a graphical
front end).  The system deduces dependencies between modules based on
patterns of data use, and modules are run concurrently where possible.


\section{Installation and Setup}
\subsection{System Requirements}

The GCAM automation system has been developed and tested on the Linux
operating system .  The code is written portably in Python, so any
Linux version
from the past few years should work.  The code
requires Python~2.7.  Due to backward incompatibilities in Python~3.X,
later versions of the Python interpreter will \emph{not} work.

Output in NetCDF format requires the NetCDF-3 libraries.  These
libraries are available from Unidata (\texttt{unidata.ucar.edu}).

The modules that run GCAM analyses are distributed separately in their
own repositories.  In addition to run them you will need to install
the code for all of the modules you wish to run.  The FY-15 demos use
the GCAM core model and the hydrology and water downscaling code.
Both of these are distributed by PNNL.  The GCAM core model is a
stand-alone program, while the hydrology and water downscaling code
requires a copy of Matlab.

\subsection{How to install}

The GCAM automation system is hosted on GitHub and can be downloaded
from:\\
\texttt{https://github.com/JGCRI/gcam-driver}\\
The release used for FY15 runs for the Foresight project is
version~0.2.  To install the code download this release and unpack it
in the directory of your choice.  If you wish to produce netCDF
outputs, make sure the netCDF-3 libraries are installed, and set the
\texttt{NETCDF\_INCDIR} environment variable to the location of the
netCDF include files, and the \texttt{NETCDF\_LIBDIR} to the location
of the netCDF libraries.  Then, from the \texttt{src/C} subdirectory
of the release, run \texttt{make mat2nc} to build the netCDF
converter.

\section{How to run}

In order to run a GCAM calculation using the automation system, you
must first prepare a configuration file describing the GCAM modules to
include.  The configuration file is divided into sections, with each
section corresponding to a module.  The section begins with the name
of the module enclosed in square brackets and consists of a series of
parameter-value pairs separated by an $=$ sign.  Parameters may be
listed in any order.  Each module has its own set of parameter
options, which are explained below.  An example configuration file is
included with the code in the \texttt{example-chn.cfg} file.

The code instantiates modules as it finds them in the configuration
file.  Because of this, a module can be excluded from a run merely by
not mentioning it in the configuration file.  Most of the modules are
independent of one another, but some of them use output from other
modules.  Trying to run a configuration that omits a module that
produces data used by another module that is included in the
configuration will cause an error.  The individual module descriptions
below note when a module has dependencies.

Once the configuration file has been prepared, the program can be run
from the command line by changing to the 
top-level gcam-driver directory and running:\\
\texttt{./gcam-driver <configfile>}\\ 


\subsection{The Global Module}
The Global module provides parameter values that will be used by multiple
other modules.  Most of these values are related to the locations of
files used to access GCAM data.  The name of this module is
\texttt{[Global]}.
\begin{description}
\item[ModelInterface] The location of the \texttt{ModelInterface.jar}
  Java file.  This will have been installed along with the GCAM core
  model.
\item[DBXMLlib] The location of the DBXML libraries.  These will also
  have been
  installed with the GCAM core model.
\item[inputdir] The location of the gcam-driver input data.  For now,
  this is always \texttt{./input-data}.  It is included as a settable
  parameter to allow for future expansion.
\item[rgnconfig] The directory containing the files that describe how
  the world is divided up into regions.  This directory path is
  relative to the directory specified for \texttt{inputdir}.  There
  are currently three to choose from:  \texttt{rgn14}, \texttt{rgn32},
  and \texttt{rgnchn}.
\end{description}


\subsection{The Hydrology Module}
The Hydrology module runs the hydrology for the future scenario and
makes the results available to other modules.  The name of the module
in the config file is \texttt{[HydroModule]}.

This module requires the Historical Hydrology module.

\begin{description}
\item[workdir] The working directory for the hydrology calculation.
  Generally, this is the location where the gcam-hydro code is
  installed.
\item[inputdir] Location of the CMIP5 model output used as input by the
  hydrology calculation.
\item[outputdir] Directory into which to write the module's output
  data.
\item[init-storage-file] File giving the initial condition for the
  water stored in the soil column.  This file is included with the
  gcam-hydro distribution, but it can be replaced if something better
  is available.
\item[logfile] File into which to write screen output from the matlab
  code.  This is mostly intended for debugging problems with the
  matlab calculation.
\item[gcm] The name of the earth system model to run the hydrology
  on.  This will be used to construct the name of both the input and
  output files.
\item[scenario] The RCP scenario to use for the calculation.  This
  will also be a component in the input and output file names.
\item[runid] The ensemble ID and start and end months for the input
  CMIP5 data.  This will be used to identify the correct input files
  in the CMIP5 data directory.
\end{description}

\subsection{The Historical Hydrology Module}
The Historical Hydrology module runs the hydrology for the historical
period.  This calculation is similar to the future hydrology described
above, but just different enough to require its own module.  The
historical input data is the same for all future scenarios, but it is
different for each ESM, as it is obtained by running the ESM over the
historical period.  The name of this module in the config file is
\texttt{[HistoricalHydroModule]}.

\begin{description}
\item[workdir] The working directory for the hydrology calculation.
  Generally, this is the location where the gcam-hydro code is
  installed.
\item[inputdir] Location of the CMIP5 model output used as input by
  the hydrology calculation.
\item[outputdir] Directory into which to write the module's output.
\item[logfile] File into which to write screen output from the
  matlab historical hydrology code.
\item[gcm] Name of the earth system model to use as input.  This
  name will be used both to locate the input data and to construct
  the file names for the output data.
\item[runid] The ensemble ID and start and end months for the input
  CMIP5 data.  This will be used to identify the correct input files
  in the CMIP5 data directory.
\end{description}

\subsection{The {GCAM} Module}
The GCAM module runs a GCAM scenario.  The name of the module in the
config file is \texttt[GcamModule] (note the lack of capitalization in
``Gcam'').

\begin{description}
\item[exe] Location of the \texttt{gcam.exe} executable.
\item[config] Location of the GCAM configuration file.  This is the
  primary input to GCAM that defines the GCAM scenario to be run.
\item[logconfig] Location of the GCAM log configuration file.  This
  file specifies the names of various GCAM log files and the logging
  level for each log.  If the log files are not needed, then it is
  safe to use the same log configuration for multiple runs.
\item[logfile] File into which to redirect the output that GCAM would
  normally send to the screen.  This output is generally intended to
  keep users running GCAM interactively apprised of the program's
  progress and can usually be discarded.
\end{description}

\subsection{The Water Disaggregation Module}

This module disaggregates the GCAM regional water demand outputs to a
half-degree grid.  It also uses the gridded result to reaggregate to
basin and regional scale (in the latter case, possibly using a
different set of region definitions than GCAM did for its output).
The module can be run using the output of the GCAM module described
above, or it can be given the name of a preexisting GCAM output
(DBXML) file.  In the latter case the GCAM module should be omitted
from the configuration.

This module requires the Hydrology module.  If the GCAM module is
included, then this module will use the output from that one.  If the
GCAM module is not included, then the location of a GCAM output file
must be supplied through the \texttt{dbxml} parameter.  In the
configuration file this model is referred to as
\texttt{[WaterDisaggregationModule]}.

\begin{description}
\item[workdir] Working directory for the water disaggregation code.
  Generally this will be the directory where the gcam-hydro code is
  installed.
\item[tempdir] Directory to use for temporary files generated by the
  module.
\item[outputdir] Directory to use for the module's final outputs.
\item[scenario] Used to name the outputs.  For consistency this should
  be the same as the scenario used in the Hydrology module, but it
  isn't required.
\item[logfile] File in which to write the screen output of the matlab
  water disaggregation code.  This information is useful only for
  debugging problems with the calculation.
\item[water-transfer] \emph{(OPTIONAL)} True/False flag indicating
  whether a basin-level water transfer should be applied.  If not
  supplied, the default is False.
\item[transfer-file] \emph{(OPTIONAL)} Name of the file specifying the
  details of the basin-level water transfer (see description below).
  This parameter is ignored (and can be omitted) if
  \texttt{water-transfer} is False; otherwise, it must be specified.
\item[dbxml] \emph{(OPTIONAL)} Name of a GCAM output file to use in
  for input.  This parameter is ignored if there is a GCAM module
  present in the configuration.  If there is not a GCAM module in the
  configuration, then this parameter must be specified.
\item[power-plant-data] \emph{(OPTIONAL)} Name of the input file
  containing the power plant data described in
  section~\ref{sec:pplant}.  If not specified, then the power plant
  data will not be used, and water usage from electrical generation
  will be disaggregated using population as a proxy (as in previous
  versions of the disaggregation calculation).
\end{description}

\subsection{The netCDF Demo Module}
The netCDF Demo module packages the output of the Hydrology and Water
Disaggregation modules into a netCDF file suitable for display in the
decision theater viewer.  This module requires both the Hydrology and
Water Disaggregation modules to be present.  In the configuration file
the module is referred to as \texttt{[NetcdfDemoModule]}.  Note that
netCDF is not capitalized in the usual way.

\begin{description}
\item[mat2nc] The location of the mat2nc program.  A copy of the C
  source for this program is included with the distribution in the
  \texttt{src/C} subdirectory.
\item[outfile] Name of the intended netCDF output file.
\item[rcp] The RCP forcing level for the GCAM scenario.  This is
  written into the global attributes for the output file to identify
  the scenario being depicted.  It does not affect the output in any
  other way.
\item[pop] The 2050 global population for the GCAM scenario.  This is
  also written into the global attributes for the output file.
\end{description}

\subsection{The ``clobber'' parameter}
In addition to the parameters described in their individual sections
above, most modules can take a parameter called ``clobber'', which
must be set to either ``True'' or ``False''.  If False, then the code
will check to see whether the anticipated results already exist.  If
they do, and if none of the upstream modules used by the module have
produced new data, then the modules calculation will not run; the
preexisting results will be used instead.  If the ``clobber''
parameter is set to True, then the module will run normally, with the
new results overwriting the preexisting results, if any.

\section{Using power plant location data}
\label{sec:pplant}
The GCAM water demand disaggregation can use data on power plant
location and type as a proxy for disaggregating water demand from
electrical generation.  Using this data is optional.  If it is not
supplied, then population will be used as the proxy for water demand
from electrical generation, as in previous versions of the
calculation.

The power plant data should be supplied as a geoJSON feature
collection.  Each power station should be a single point feature in
the list.  Each station should have a ``properties'' member, with a
value that is itself a JSON object.  The ``properties'' object must
have the following members:
\begin{description}
\item[fuel] String giving the fuel type.  Three types are currently
  recognized:  ``Coal'', ``Nuclear'', and ``Gas''.
\item[capacity] String giving the nameplate capacity of the plant in
  MWe.  The unit ``MWe'' may be included in the string or omitted.
\end{description}
Object attributes beyond the required ones are permitted, and will be
ignored by the disaggregation code.  An example set of inputs is
provided in the input directory in the file
\texttt{power-plants.geojson}.

If the power plant data is provided, then water withdrawals will be
computed for each plant by multiplying the nameplate capacity by a
fuel-specific water intensity factor.  The water intensity was chosen
by selecting the generation and cooling technologies representative of
the type of plants expected to be built over the next few decades:
\begin{center}
\begin{tabular}{|l|l|r|}
  \hline
  fuel & assumed type & water intensity (m$^3$/MWh) \\\hline
  Coal & steam, recirculating &  3.80  \\
  Gas  & combined-cycle, recirculating &  0.96 \\
  Nuclear & steam, recirculating &  4.17 \\\hline
\end{tabular}
\end{center}
This procedure could readily be extended to factor in explicitly the
generation and cooling technologies, if a data set including that
information were available.

These values are converted to a grid by summing the water withdrawals
of the plants in each grid cell.  This grid will form the
disaggregation proxy for the water calculation.  In practice, there
are usually a few regions that no power plants within their borders.
This typically happens with in sub-national regions, which are often
part of a larger electrical grid and therefore may supply their power
from generating capacity located outside their borders.  In these
cases we revert to the population-proxy to allocate water withdrawals
within the affected regions.

The power plant proxy is asumed to be constant over the course of the
simulation.  This has the effect of causing new capacity produced in
GCAM simulations to be allocated to locations with existing capacity,
and in fixed proportions.  Modeling the siting and size distribution
of new power stations is beyond GCAM's scope; however, if a model for
those properties were available, GCAM could be coupled to it to
produce more realistic spatial distributions in the future.

\appendix
\section{Technical Reference}
Technical details for the python modules that make up the GCAM
automation system can be obtained by using the \texttt{pydoc} command
included with most python distributions.  For this to work the
PYTHONPATH environment variable must include the \texttt{src}
directory of the \texttt{gcam-driver} installation.  For csh-type
command shells this can be accomplished by running:\\ 
\texttt{setenv PYTHONPATH /path/to/gcam-driver/src}\\
For bourne-type shells the command is:\\
\texttt{export PYTHONPATH=/path/to/gcam-driver/src}.

The information given in the pydoc documentation pertains mostly to
the internals of the GCAM automation system, including function
arguments and return values and descriptions of internal data
structures.  It will therefore be useful primarily to users interested
in extending the system.

\end{document}
